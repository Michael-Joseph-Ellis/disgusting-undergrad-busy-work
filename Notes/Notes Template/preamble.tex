%%%%%%%%%%%%%%%%%%%%%%%
%       Imports       %
%%%%%%%%%%%%%%%%%%%%%%%

% Imports for document styling and formatting
\usepackage[tmargin=2cm,rmargin=1in,lmargin=1in,margin=0.85in,bmargin=2cm,footskip=.2in]{geometry} % for margins
\usepackage{xcolor} % for colors
\usepackage{bookmark} % for bookmarks
\usepackage{comment} % enables to use of multiline comments (\ifx \fi)
\usepackage{nameref} % for names
\usepackage{transparent} % for transparency
\usepackage[makeroom]{cancel} % for canceling terms
\usepackage{authblk} % for author affiliations
\usepackage{import} % for importing files
\usepackage{pdfpages} % for including pdfs
\usepackage{titletoc} % for table of contents 
\usepackage{titlesec} % For `titlecontents` an`d TOC customization
\usepackage{pgf} % for tikz
\usepackage[utf8]{inputenc} % for input encoding
\usepackage[T1]{fontenc} % for font encoding
\usepackage{lmodern} % for modern fonts 
\usepackage{newtxtext} % for text fonts

\setlength{\parindent}{1cm} % for paragraph indentation
\newcounter{mylabelcounter} % to create a counter for labels

\newcommand{\incfig}[1]
{
	\def\svgwidth{\columnwidth}
    \import{./figures/}{#1.pdf_tex}
} % for importing svg files
	
\usepackage{hyperref} % for hyperlinks and references

\hypersetup{
	pdftitle = {Title},
	colorlinks = true, linkcolor = doc!50!black, citecolor = doc!50!black, urlcolor = doc!50!black,
	bookmarksnumbered = true, bookmarksopen = true
} % hyperlink setup and metadata for the pdf

% Imports for Math formatting and symbols
\usepackage{amsmath, amssymb, amsthm, amsfonts, mathtools} % for math
\usepackage[varbb]{newpxmath} % for math fonts
\usepackage{xfrac} % for slanted fractions
\renewcommand\qedsymbol{$\blacksquare$} % for qed symbol

% Imports for images and diagrams 
\usepackage{graphicx} % for images

% Imports for lists, tables, columns, and boxes 
\usepackage{enumitem} % for lists
\usepackage{multicol, array} % for columns and arrays
\usepackage{varwidth} % for boxes
\usepackage[most, many, breakable]{tcolorbox} % for boxes
\tcbuselibrary{skins} % for tcolorbox libraries
\usepackage{framed} % for boxes 

% Imports for code and algorithms
\usepackage{etoolbox} % for if statements
\usepackage{xifthen} % for if statements
\usepackage[ruled,vlined,linesnumbered]{algorithm2e} % for algorithms
\SetCommentSty{mycommfont} % for comments in algorithms
\newcommand\mycommfont[1]{\footnotesize\ttfamily\textcolor{blue}{#1}} % for comments in algorithms

% Imports for references and hyperlinks
\usepackage{hyperref,theoremref} % for hyperlinks and references

% Tiks
\usepackage{tikzsymbols} % for symbols    
\usepackage{tikz} % for diagrams
\usepackage{tikz-cd} % for commutative diagrams
\usetikzlibrary{arrows,calc,shadows.blur} % for tikz libraries

\tikzset{
	symbol/.style = 
    {
			draw = none,
			every to/.append style = 
            {
					edge node = 
                    { 
                        node [sloped, allow upside down, auto = false]
                        {
                            $#1$
                        }
                    }
            }
    }
} % for symbols

%%%%%%%%%%%%%%%%%%%%%%%
%        Colors       %
%%%%%%%%%%%%%%%%%%%%%%%

\definecolor{my_red}{HTML}{bd0000} % dark red
\definecolor{my_blue}{HTML}{001589} % dark blue
\definecolor{my_green}{HTML}{033b18} % dark green
\definecolor{my_purple}{HTML}{4c0088} % dark purple
\definecolor{my_gray}{HTML}{565656} % dark gray
\definecolor{my_yellow}{HTML}{b9a900} % dark yellow 
\definecolor{my_black}{HTML}{000000} % black

\definecolor{theorem_BG}{HTML}{F2F2F9} % light gray
\definecolor{theorem_F}{HTML}{00007B} % dark blue
\definecolor{corollary_BG}{HTML}{4c0088} % dark purple
\definecolor{corollary_F}{HTML}{000000} % black
\definecolor{lemma_BG}{HTML}{196800} % dark green
\definecolor{lemma_F}{HTML}{00091e} % deep dark blue
\definecolor{proposition_BG}{HTML}{005fe8} % dark blue
\definecolor{proposition_F}{HTML}{004246} % dark teal
\definecolor{exercise_BG}{HTML}{f6fcfc} % blank white 
\definecolor{exercise_F}{HTML}{417576} % deep teal
\definecolor{example_BG}{HTML}{f9f9f9} % light gray
\definecolor{example_F}{HTML}{000000} % black
\definecolor{example_TI}{HTML}{000000} % black
\definecolor{solution_BG}{HTML}{f9f9f9} % light gray 
\definecolor{solution_F}{HTML}{000000} % black 
\definecolor{solution_TI}{HTML}{000000} % black 
\definecolor{doc}{HTML}{009fdf} % light blue
\definecolor{TOC_COLOR1}{HTML}{6000c6} % Clemson Purple
\definecolor{TOC_COLOR2}{HTML}{ffa716} % Clemson Orange

%%%%%%%%%%%%%%%%%%%%%%%
%     TCOLORBOXES     %
%%%%%%%%%%%%%%%%%%%%%%%

% The one thing that is confusing about these boxes including their names/commands
% is that they are not consistent with the actual section/chapter numbers. 
% For example, the section_theorem box is not numbered as Theorem 1.1 but as Theorem 1.1.1 
% however the chapter_theorem box is numbered as Theorem 1.1.
% In other words, section_theorem is numbered via subsection while chapter_theorem is numbered
% via section.  
% When using these commands, use chapter_box under sections and section_box under subsections.

\makeatletter
% Theorem Boxes % 
% Section Theorem 
\newtcbtheorem[number within=section]{section_theorem}{Theorem}
{
	enhanced,
	breakable,
	colback = theorem_BG,
	frame hidden,
	boxrule = 0sp,
	borderline west = {2pt}{0pt}{theorem_F},
	sharp corners,
	detach title,
	before upper = \tcbtitle\par\smallskip,
	coltitle = theorem_F,
	fonttitle = \bfseries\sffamily,
	description font = \mdseries,
	separator sign none,
	segmentation style={solid, theorem_F}
}
{theorem}

% Chapter Theorem
\newtcbtheorem[number within=chapter]{chapter_theorem}{Theorem}
{
	enhanced,
	breakable,
	colback = theorem_BG,
	frame hidden,
	boxrule = 0sp,
	borderline west = {2pt}{0pt}{theorem_F},
	sharp corners,
	detach title,
	before upper = \tcbtitle\par\smallskip,
	coltitle = theorem_F,
	fonttitle = \bfseries\sffamily,
	description font = \mdseries,
	separator sign none,
	segmentation style={solid, theorem_F}
}
{theorem}

% Corollery Boxes % 
% Section Corollary
\newtcbtheorem[number within=section]{section_corollary}{Corollary}
{
	enhanced,
	breakable,
	colback = corollary_BG!10,
	frame hidden,
	boxrule = 0sp,
	borderline west = {2pt}{0pt}{my_purple!85!corollary_F},
	sharp corners,
	detach title,
	before upper = \tcbtitle\par\smallskip,
	coltitle = corollary_BG!85!corollary_F,
	fonttitle = \bfseries\sffamily,
	description font = \mdseries,
	separator sign none,
	segmentation style={solid, corollary_BG!85!corollary_F}
}
{corollary}

% Chapter Corollary
\newtcbtheorem[number within=chapter]{chapter_corollary}{Corollary}
{
	enhanced,
	breakable,
	colback = corollary_BG!10,
	frame hidden,
	boxrule = 0sp,
	borderline west = {2pt}{0pt}{corollary_BG!85!corollary_F},
	sharp corners,
	detach title,
	before upper = \tcbtitle\par\smallskip,
	coltitle = corollary_BG!85!corollary_F,
	fonttitle = \bfseries\sffamily,
	description font = \mdseries,
	separator sign none,
	segmentation style={solid, corollary_BG!85!corollary_F}
}
{corollary}

% Lemma Boxes %
% Section Lemma
\newtcbtheorem[number within=section]{section_lemma}{Lemma}
{
	enhanced,
	breakable,
	colback = lemma_BG!10,
	frame hidden,
	boxrule = 0sp,
	borderline west = {2pt}{0pt}{lemma_F},
	sharp corners,
	detach title,
	before upper = \tcbtitle\par\smallskip,
	coltitle = lemma_F,
	fonttitle = \bfseries\sffamily,
	description font = \mdseries,
	separator sign none,
	segmentation style={solid, lemma_F}
}
{lemma}

% Chapter Lemma
\newtcbtheorem[number within=chapter]{chapter_lemma}{Lemma}
{
	enhanced,
	breakable,
	colback = lemma_BG!10,
	frame hidden,
	boxrule = 0sp,
	borderline west = {2pt}{0pt}{lemma_F},
	sharp corners,
	detach title,
	before upper = \tcbtitle\par\smallskip,
	coltitle = lemma_F,
	fonttitle = \bfseries\sffamily,
	description font = \mdseries,
	separator sign none,
	segmentation style={solid, lemma_F}
}
{lemma}

% Proposition Boxes %
% Section Proposition
\newtcbtheorem[number within=section]{section_proposition}{Proposition}
{
	enhanced,
	breakable,
	colback = proposition_BG!10,
	frame hidden,
	boxrule = 0sp,
	borderline west = {2pt}{0pt}{proposition_F},
	sharp corners,
	detach title,
	before upper = \tcbtitle\par\smallskip,
	coltitle = proposition_F,
	fonttitle = \bfseries\sffamily,
	description font = \mdseries,
	separator sign none,
	segmentation style={solid, proposition_F}
}
{proposition}

% Chapter Proposition
\newtcbtheorem[number within=chapter]{chapter_proposition}{Proposition}
{
	enhanced,
	breakable,
	colback = proposition_BG!10,
	frame hidden,
	boxrule = 0sp,
	borderline west = {2pt}{0pt}{proposition_F},
	sharp corners,
	detach title,
	before upper = \tcbtitle\par\smallskip,
	coltitle = proposition_F,
	fonttitle = \bfseries\sffamily,
	description font = \mdseries,
	separator sign none,
	segmentation style={solid, proposition_F}
}
{proposition}

% Claim Boxes %
% Section Claim 
\newtcbtheorem[number within=section]{section_claim}{Claim}
{
	enhanced,
	breakable,
	colback = my_red!10,
	frame hidden,
	boxrule = 0sp,
	borderline west = {2pt}{0pt}{my_red},
	sharp corners,
	detach title,
	before upper = \tcbtitle\par\smallskip,
	coltitle = my_red!85!my_black,
	fonttitle = \bfseries\sffamily,
	description font = \mdseries,
	separator sign none,
	segmentation style={solid, my_red!85!my_black}
}
{claim}

% Chapter Claim
\newtcbtheorem[number within=chapter]{chapter_claim}{Claim}
{
	enhanced,
	breakable,
	colback = my_red!10,
	frame hidden,
	boxrule = 0sp,
	borderline west = {2pt}{0pt}{my_red},
	sharp corners,
	detach title,
	before upper = \tcbtitle\par\smallskip,
	coltitle = my_red!85!my_black,
	fonttitle = \bfseries\sffamily,
	description font = \mdseries,
	separator sign none,
	segmentation style={solid, my_red!85!my_black}
}
{claim}

% Exercise Boxes %
% Section Exercise  
\newtcbtheorem[number within=section]{section_exercise}{Exercise}
{
	enhanced,
	breakable,
	colback = exercise_BG,
	frame hidden,
	boxrule = 0sp,
	borderline west = {2pt}{0pt}{exercise_F},
	sharp corners,
	detach title,
	before upper = \tcbtitle\par\smallskip,
	coltitle = exercise_F,
	fonttitle = \bfseries\sffamily,
	description font = \mdseries,
	separator sign none,
	segmentation style={solid, exercise_F},
}
{exercise}

% Chapter Exercise 
\newtcbtheorem[number within=chapter]{chapter_exercise}{Exercise}
{
	enhanced,
	breakable,
	colback = exercise_BG,
	frame hidden,
	boxrule = 0sp,
	borderline west = {2pt}{0pt}{exercise_F},
	sharp corners,
	detach title,
	before upper = \tcbtitle\par\smallskip,
	coltitle = exercise_F,
	fonttitle = \bfseries\sffamily,
	description font = \mdseries,
	separator sign none,
	segmentation style={solid, exercise_F},
}
{exercise}

% Example Boxes %
% Section Example 
\newtcbtheorem[number within=section]{section_example}{Example}
{
	colback = example_BG,
	breakable,
	colframe = example_F,
	coltitle = example_TI,
	boxrule = 1pt,
	sharp corners,
	detach title,
	before upper=\tcbtitle\par\smallskip,
	fonttitle = \bfseries,
	description font = \mdseries,
	separator sign none,
	description delimiters parenthesis
}
{example}

% Chapter Example
\newtcbtheorem[number within=chapter]{chapter_example}{Example}
{
	colback = example_BG,
	breakable,
	colframe = example_F,
	coltitle = example_TI,
	boxrule = 1pt,
	sharp corners,
	detach title,
	before upper=\tcbtitle\par\smallskip,
	fonttitle = \bfseries,
	description font = \mdseries,
	separator sign none,
	description delimiters parenthesis
}
{example}

% Definition Boxes %
% Section Definition
\newtcbtheorem[number within=section]{section_definition}{Definition}
{
    enhanced,
	before skip = 2mm,
    after skip = 2mm, 
    colback = red!5,
    colframe = red!80!black,
    boxrule = 0.5mm,
	attach boxed title to top left = 
    {
        xshift = 1cm,
        yshift* = 1mm-\tcboxedtitleheight,
    }, 
    varwidth boxed title* = -3cm,
	boxed title style = 
    {
        frame code = 
        {
					\path[fill = tcbcolback]
                    ([yshift = -1mm, xshift = -1mm]frame.north west)
					arc[start angle = 0, end angle = 180, radius = 1mm]
					([yshift = -1mm, xshift = 1mm]frame.north east)
					arc[start angle = 180, end angle = 0, radius = 1mm];
					\path[left color = tcbcolback!60!black, right color = tcbcolback!60!black,
						middle color = tcbcolback!80!black]
					([xshift = -2mm]frame.north west)-- 
                    ([xshift = 2mm]frame.north east)[rounded corners = 1mm]-- 
                    ([xshift = 1mm, yshift = -1mm]frame.north east)--
					(frame.south east)-- 
                    (frame.south west)--
					([xshift = -1mm, yshift = -1mm]frame.north west)[sharp corners]-- 
                    cycle;
        },
        interior engine = empty,
    },
	fonttitle = \bfseries,
	title = {#2},
    #1
}{definition}

% Chapter Definition
\newtcbtheorem[number within=chapter]{chapter_definition}{Definition}
{
    enhanced,
	before skip = 2mm,
    after skip = 2mm, 
    colback = red!5,
    colframe = red!80!black,
    boxrule = 0.5mm,
	attach boxed title to top left = 
    {
        xshift = 1cm,
        yshift* = 1mm-\tcboxedtitleheight,
    }, 
    varwidth boxed title* = -3cm,
	boxed title style = 
    {
        frame code = 
        {
					\path[fill = tcbcolback]
                    ([yshift = -1mm, xshift = -1mm]frame.north west)
					arc[start angle = 0, end angle = 180, radius = 1mm]
					([yshift = -1mm, xshift = 1mm]frame.north east)
					arc[start angle = 180, end angle = 0, radius = 1mm];
					\path[left color = tcbcolback!60!black, right color = tcbcolback!60!black,
						middle color = tcbcolback!80!black]
					([xshift = -2mm]frame.north west)-- 
                    ([xshift = 2mm]frame.north east)[rounded corners = 1mm]-- 
                    ([xshift = 1mm, yshift = -1mm]frame.north east)--
					(frame.south east)-- 
                    (frame.south west)--
					([xshift = -1mm, yshift = -1mm]frame.north west)[sharp corners]-- 
                    cycle;
        },
        interior engine = empty,
    },
	fonttitle = \bfseries,
	title = {#2},
    #1
}{definition}

% Question Boxes %
% Section Question
\newtcbtheorem[number within=section]{section_question}{Question}
{
    enhanced,
	before skip = 2mm,
    after skip = 2mm, 
    colback = my_green!5,
    colframe = my_green!80!my_black,
    boxrule = 0.5mm,
	attach boxed title to top left = 
    {
        xshift = 1cm,
        yshift* = 1mm-\tcboxedtitleheight,
    }, 
    varwidth boxed title* = -3cm,
	boxed title style = 
    {
        frame code = 
        {
					\path[fill = my_green!2!my_black]
                    ([yshift = -1mm, xshift = -1mm]frame.north west)
					arc[start angle = 0, end angle = 180, radius = 1mm]
					([yshift = -1mm, xshift = 1mm]frame.north east)
					arc[start angle = 180, end angle = 0, radius = 1mm];
					\path[left color = my_green!60!black, right color = my_green!60!black,
						middle color = my_green!80!black]
					([xshift = -2mm]frame.north west)-- 
                    ([xshift = 2mm]frame.north east)[rounded corners = 1mm]-- 
                    ([xshift = 1mm, yshift = -1mm]frame.north east)--
					(frame.south east)-- 
                    (frame.south west)--
					([xshift = -1mm, yshift = -1mm]frame.north west)[sharp corners]-- 
                    cycle;
        },
        interior engine = empty,
    },
	fonttitle = \bfseries,
	title = {#2},
    #1
}{question}

% Chapter Question 
\newtcbtheorem[number within=chapter]{chapter_question}{Question}
{
    enhanced,
	before skip = 2mm,
    after skip = 2mm, 
    colback = my_green!5,
    colframe = my_green!80!my_black,
    boxrule = 0.5mm,
	attach boxed title to top left = 
    {
        xshift = 1cm,
        yshift* = 1mm-\tcboxedtitleheight,
    }, 
    varwidth boxed title* = -3cm,
	boxed title style = 
    {
        frame code = 
        {
					\path[fill = my_green!2!my_black]
                    ([yshift = -1mm, xshift = -1mm]frame.north west)
					arc[start angle = 0, end angle = 180, radius = 1mm]
					([yshift = -1mm, xshift = 1mm]frame.north east)
					arc[start angle = 180, end angle = 0, radius = 1mm];
					\path[left color = my_green!60!black, right color = my_green!60!black,
						middle color = my_green!80!black]
					([xshift = -2mm]frame.north west)-- 
                    ([xshift = 2mm]frame.north east)[rounded corners = 1mm]-- 
                    ([xshift = 1mm, yshift = -1mm]frame.north east)--
					(frame.south east)-- 
                    (frame.south west)--
					([xshift = -1mm, yshift = -1mm]frame.north west)[sharp corners]-- 
                    cycle;
        },
        interior engine = empty,
    },
	fonttitle = \bfseries,
	title = {#2},
    #1
}{question}

% Solution Boxes % 
% Section Solution
\newtcbtheorem[number within=section]{section_solution}{Solution}
{
	colback = solution_BG,
	breakable,
	colframe = solution_F,
	coltitle = solution_TI,
	boxrule = 1pt,
	sharp corners,
	detach title,
	before upper=\tcbtitle\par\smallskip,
	fonttitle = \bfseries,
	description font = \mdseries,
	separator sign none,
	description delimiters parenthesis
}
{solution}

% Chapter Solution 
\newtcbtheorem[number within=chapter]{chapter_solution}{Solution}
{
	colback = solution_BG,
	breakable,
	colframe = solution_F,
	coltitle = solution_TI,
	boxrule = 1pt,
	sharp corners,
	detach title,
	before upper=\tcbtitle\par\smallskip,
	fonttitle = \bfseries,
	description font = \mdseries,
	separator sign none,
	description delimiters parenthesis
}
{solution}

% Note Bos % 
\newtcolorbox{note}[1][]{
	enhanced jigsaw,
	colback = gray!20!white,
	colframe = gray!80!black,
	size = small,
	boxrule = 1pt,
	title = \textbf{Note:-},
	halign title = flush center,
	coltitle = black,
	breakable,
	drop shadow = black!50!white,
	attach boxed title to top left = 
	{
		xshift = 1cm, yshift = -\tcboxedtitleheight/2,yshifttext = -\tcboxedtitleheight/2
	},
	minipage boxed title = 1.5cm,
	boxed title style = 
	{
			colback = white,
			size = fbox,
			boxrule = 1pt,
			boxsep = 2pt,
			underlay = {
					\coordinate(dotA) at ($(interior.west) + (-0.5pt, 0)$);
					\coordinate(dotB) at ($(interior.east) + (0.5pt, 0)$);
					\begin{scope}
						\clip(interior.north west) rectangle ([xshift = 3ex]interior.east);
						\filldraw[white, blur shadow = {shadow opacity = 60, shadow yshift = -.75ex}, rounded corners = 2pt] (interior.north west) rectangle (interior.south east);
					\end{scope}
					\begin{scope}[gray!80!black]
						\fill (dotA) circle (2pt);
						\fill (dotB) circle (2pt);
					\end{scope}
				},
		},
	#1,
}

%%%%%%%%%%%%%%%%%%%%%%%
%       Commands      %
%\command{title}{desc}%
%%%%%%%%%%%%%%%%%%%%%%%

% Theorem Commands %
\newcommand{\sectiontheorem}[2]
{
	\begin{section_theorem}{#1}{}#2\end{section_theorem}
}

\newcommand{\chaptertheorem}[2]
{
	\begin{chapter_theorem}{#1}{}#2\end{chapter_theorem}
}

% Corollary Commands %
\newcommand{\sectioncorollary}[2]
{
	\begin{section_corollary}{#1}{}#2\end{section_corollary}
}

\newcommand{\chaptercorollary}[2]
{
	\begin{chapter_corollary}{#1}{}#2\end{chapter_corollary}
}

% Lemma Commands %
\newcommand{\sectionlemma}[2]
{
	\begin{section_lemma}{#1}{}#2\end{section_lemma}
}

\newcommand{\chapterlemma}[2]
{
	\begin{chapter_lemma}{#1}{}#2\end{chapter_lemma}
}

% Proposition Commands %
\newcommand{\sectionproposition}[2]
{
	\begin{section_proposition}{#1}{}#2\end{section_proposition}
}

\newcommand{\chapterproposition}[2]
{
	\begin{chapter_proposition}{#1}{}#2\end{chapter_proposition}
}

% Claim Commands %
\newcommand{\sectionclaim}[2]
{
	\begin{section_claim}{#1}{}#2\end{section_claim}
}

\newcommand{\chapterclaim}[2]
{
	\begin{chapter_claim}{#1}{}#2\end{chapter_claim}
}

% Exercise Commands %
\newcommand{\sectionexercise}[2]
{
	\begin{section_exercise}{#1}{}#2\end{section_exercise}
}

\newcommand{\chapterexercise}[2]
{
	\begin{chapter_exercise}{#1}{}#2\end{chapter_exercise}
}

% Example Commands %
\newcommand{\sectionexample}[2]
{
	\begin{section_example}{#1}{}#2\end{section_example}
}

\newcommand{\chapterexample}[2]
{
	\begin{chapter_example}{#1}{}#2\end{chapter_example}
}

% Definition Commands %
\newcommand{\sectiondefinition}[2]
{
	\begin{section_definition}{#1}{}#2\end{section_definition}
}

\newcommand{\chapterdefinition}[2]
{
	\begin{chapter_definition}{#1}{}#2\end{chapter_definition}
}

% Question Commands %
\newcommand{\sectionquestion}[2]
{
	\begin{section_question}{#1}{}#2\end{section_question}
}

\newcommand{\chapterquestion}[2]
{
	\begin{chapter_question}{#1}{}#2\end{chapter_question}
}

% Solution Commands %
\newcommand{\sectionsolution}[2]
{
	\begin{section_solution}{#1}{}#2\end{section_solution}
}

\newcommand{\chaptersolution}[2]
{
	\begin{chapter_solution}{#1}{}#2\end{chapter_solution}
}

% Note Commands %
% Ill do this later since its buggy

% My Proof Command %
\newenvironment{myproof}[1][\proofname]
{
	\proof[\bfseries #1: ]
}{\endproof}

% My Claim Command % 
\newcommand{\myclaim}[2]{\begin{my_claim}[#1]#2\end{my_claim}}
\newenvironment{my_claim}[1][\claimname]
{
	\proof[\bfseries #1: ]
}{}

%%%%%%%%%%%%%%%%%%%
% Random Commands %
%%%%%%%%%%%%%%%%%%%

\newcommand*\circled[1]
{
	\tikz[baseline = (char.base)]
	{
		\node[shape = circle, draw, inner sep = 1pt] (char) {#1};
	}
} % to circle something \circled{something}

\newcommand\getcurrentref[1]
{
	\ifnumequal{\value{#1}}{0}
	{??}
	{\the\value{#1}}
} % to get the current reference. ex: \getcurrentref{subsection}

\newcommand{\getCurrentSectionNumber}
{
    \getcurrentref{section}
} % to get the current section number	

\newcommand{\getCurrentChapterNumber}
{
    \getcurrentref{chapter}
} % to get the current chapter number

\newcommand{\setword}[2]
{
	\phantomsection#1\def\@currentlabel{\unexpanded{#1}}\label{#2} 
} % to set a word for later reference \setword{word}{label} call later with \ref{label}

% Partial Derivatives
\newsavebox\diffdbox\newcommand{\slantedromand}{{\mathpaletta\makesl{d}}} % to make a slanted d
\newcommand{\makesl}[2]
{
    \begingroup
    \sbox{\diffdbox}{$\mathsurround = 0pt#1\mathrm{#2}$}
    \pdfsave\pdfsetmatrix{1 0 0.2 1}
    \rlap{\usebox{\diffdbox}}
    \pdfrestore\hskip\wd\diffdbox\endgroup
} % to make a slanted d

% The following is an example of the \makesl command 
% \[
%     \int_0^1 8x^2 \makesl{d}x = \left[ \frac{8x^3}{3} \right]_0^1 = \frac{8}{3}.
% \]

\providecommand*{\pdv}[3][]{\frac{\partial^{#1}#2}{\partial#3^{#1}}}

% Below is an example of the \pdv command
% 1. First-order partial derivative:
% \[
%   \pdv{f}{x}
% \]
% 
% 2. Second-order partial derivative:
% \[
%   \pdv[2]{f}{x}
% \]
% 
% 3. Mixed partial derivatives:
% \[
%   \pdv[2]{f}{x \partial y} = \frac{\partial^2 f}{\partial x \partial y}
% \]
% 
% 4. Higher-order partial derivatives:
% \[
%   \pdv[3]{f}{z}
% \]

% save them in case they're every wanted
\let\oldleq\leq\let\oldgeq\geq\renewcommand{\leq}{\leqslant}\renewcommand{\geq}{\geqslant}
% Below is an example of the \leq and \geq commands
% $a \leq b \quad \text{outputs: } a \leqslant b$

%%%%%%%%%%%%%%%%%%%%%
% TABLE OF CONTENTS %
%%%%%%%%%%%%%%%%%%%%%
% FIX THIS LATER

\contentsmargin{2pc}
\titlecontents{chapter}[6.7pc]
{
    \addvspace{30pt}
    \raisebox{-0.2cm}{
        \begin{tikzpicture}[remember picture, overlay]
            \draw[fill=TOC_COLOR1, draw=TOC_COLOR1] (-7, -.1) rectangle (-0.9, .5);
            \pgftext[left, x=-3.5cm, y=0.18cm]{\color{white}{\Large\scshape\textbf{Chapter \thecontentslabel}}};
        \end{tikzpicture}
    }
    \color{TOC_COLOR1}\large\scshape\bfseries
}
{}
{}
{
    \;\titlerule\;\large\scshape\bfseries Page \thecontentspage\raisebox{-0.1cm}
	{
        \begin{tikzpicture}[remember picture, overlay]
            \draw[fill=TOC_COLOR2, draw=TOC_COLOR2] (2pt, 0) rectangle (4, 0.1pt);
        \end{tikzpicture}
    }
}
\titlecontents{section}[3.7pc]
{\addvspace{2pt}}
{\contentslabel[\thecontentslabel]{2pc}}
{}
{\hfill\small \thecontentspage}
[]
\titlecontents{subsection}[6.7pc]
{\addvspace{2pt}\small}
{\contentslabel[\thecontentslabel]{2pc}}
{}
{\ --- \small\thecontentspage}
[]

\renewcommand{\tableofcontents}
{
    \chapter*{
        \vspace*{-20\p@}
        \raisebox{-0.2cm}{
            \begin{tikzpicture}[remember picture, overlay]
                \pgftext[right, x=15cm, y=0.2cm]{\color{TOC_COLOR2}\Huge\scshape\bfseries \contentsname};
                \draw[fill=TOC_COLOR2, draw=TOC_COLOR2] (13,-.75) rectangle (20, 1);
                \clip (13, -.75) rectangle (20,1);
                \pgftext[right, x=15cm, y=0.2cm]{\color{white}\Huge\scshape\bfseries \contentsname};
            \end{tikzpicture}
        }
    }
    \@starttoc{toc}
}

\makeatother