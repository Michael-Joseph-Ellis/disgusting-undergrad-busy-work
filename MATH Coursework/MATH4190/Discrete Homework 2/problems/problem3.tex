% Preamble needs:
% \usepackage{amsmath,amssymb,amsthm}

\section*{3. Prove the following statements}

\begin{theorem}[a]
The square of an odd integer is odd.
\end{theorem}
\begin{proof}
Let $n$ be an odd integer. Then there exists $k\in\mathbb{Z}$ such that $n=2k+1$.
Hence
\[
n^{2}=(2k+1)^{2}=4k^{2}+4k+1=2\bigl(2k^{2}+2k\bigr)+1,
\]
which is of the form $2t+1$ for some integer $t$. Therefore $n^{2}$ is odd.
\end{proof}

\begin{theorem}[b]
The difference between consecutive perfect squares is odd.
\end{theorem}
\begin{proof}
For any $n\in\mathbb{Z}$ we have
\[
(n+1)^{2}-n^{2}=n^{2}+2n+1-n^{2}=2n+1,
\]
which is odd. Thus the difference of consecutive squares is always odd.
\end{proof}

\begin{theorem}[c]
An integer is odd if and only if it is the sum of two consecutive integers.
\end{theorem}
\begin{proof}
($\Rightarrow$) Suppose $n$ is odd. Then $n=2k+1$ for some $k\in\mathbb{Z}$, and
\[
n=2k+1=k+(k+1),
\]
which is a sum of consecutive integers.

($\Leftarrow$) Conversely, suppose $n$ is the sum of two consecutive integers. Then
$n=m+(m+1)=2m+1$, which is odd. This proves the equivalence.
\end{proof}

\begin{theorem}[d]
If $n$ is a perfect square, then $n+2$ is not a perfect square.
\end{theorem}
\begin{proof}
Let $n=k^2$ with $k\in\mathbb{Z}_{\ge 0}$.

If $k=0$, then $n+2=2$ is not a square. Now assume $k\ge 1$.
Compare $k^2+2$ with the next square $(k+1)^2$:
\[
(k+1)^2 - (k^2+2) = (k^2+2k+1) - (k^2+2) = 2k-1.
\]
Since $k\ge 1$, we have $2k-1\ge 1$, so
\[
k^2 \;<\; k^2+2 \;<\; (k+1)^2.
\]
In words: $k^2+2$ lies strictly between two consecutive squares, namely
$k^2$ and $(k+1)^2$. There is no square between consecutive squares, so
$k^2+2$ cannot itself be a square. Therefore $n+2$ is not a perfect square.
