% ===== Problem 6 fragment (include with \input) =====
\section*{Problem 6}
\textbf{Problem.} How many ways are there to distribute five balls into seven boxes if each box may contain \emph{at most one} ball?

\medskip
\textbf{(a) Balls and boxes labeled.}
With capacity 1, each labeled ball must go to a distinct labeled box. This is an \emph{injection}
from a 5-element set to a 7-element set, counted by permutations:
\[
P(7,5)=7\cdot 6\cdot 5\cdot 4\cdot 3=\frac{7!}{2!}=2520.
\]

\medskip
\textbf{(b) Balls labeled, boxes unlabeled.}
Distinct balls, \emph{identical} boxes, capacity 1. Boxes are indistinguishable and at most one ball per box,
so exactly 5 boxes are used and each box contains one ball. This is a partition of the 5 distinct balls
into 5 unlabeled singletons, counted by the Stirling number of the second kind \(S(5,5)=1\).
Hence the answer is \(\boxed{1}\).

\medskip
\textbf{(c) Balls unlabeled, boxes labeled.}
Identical balls, \emph{distinct} boxes, capacity 1. We simply choose which 5 boxes receive a ball:
\[
\binom{7}{5}=21.
\]

\medskip
\textbf{(d) Balls unlabeled, boxes unlabeled.}
Identical balls, \emph{identical} boxes, capacity 1. We only care about how many boxes are occupied.
Exactly 5 boxes are occupied with one ball each, which corresponds to the single integer partition
\(1+1+1+1+1\). Therefore the answer is \(\boxed{1}\).
