\section*{8. Example with $f\circ g$ bijective but $g$ not onto and $f$ not one-to-one}

Let $A=\{1,2\}$, $B=\{1,2,3\}$, $C=\{1,2\}$.

Define $g:A\to B$ by $g(1)=1$, $g(2)=2$. Then $g$ is not onto (it misses $3$).

Define $f:B\to C$ by $f(1)=1$, $f(2)=2$, $f(3)=2$. Then $f$ is not one-to-one
(since $f(2)=f(3)$).

Composition: $(f\circ g)(1)=f(1)=1$ and $(f\circ g)(2)=f(2)=2$,
so $f\circ g:A\to C$ is the identity on $\{1,2\}$, hence a bijection.
