\section*{4. From $A\subseteq B$ and $B\subseteq C$ to $A\subseteq C$}

\subsection*{(a) Element-chasing proof}
\begin{proof}
Assume $A\subseteq B$ and $B\subseteq C$. Let $x\in A$ be arbitrary.
Then $x\in B$, and therefore $x\in C$. Since every element of $A$ lies in $C$,
we have $A\subseteq C$.
\end{proof}

\subsection*{(b) Truth-table proof}
Let $P$ denote “$x\in A$,” $Q$ denote “$x\in B$,” and $R$ denote “$x\in C$.”
Then $A\subseteq B$ is $P\to Q$ and $B\subseteq C$ is $Q\to R$.
We verify $(P\to Q)\land(Q\to R)\to(P\to R)$ is always true.

\[
\begin{array}{c c c|c c|c}
P & Q & R & P\to Q & Q\to R & (P\to Q)\land(Q\to R)\to(P\to R)\\ \hline
T & T & T & T & T & T\\
T & T & F & T & F & T\\
T & F & T & F & T & T\\
T & F & F & F & T & T\\
F & T & T & T & T & T\\
F & T & F & T & F & T\\
F & F & T & T & T & T\\
F & F & F & T & T & T
\end{array}
\]
Since the final column is $T$ in every row, the statement
\[
(P\to Q)\land(Q\to R)\to(P\to R)
\]
is true for all possible truth values of $P,Q,R$. Interpreting
$P,Q,R$ as “$x\in A$,” “$x\in B$,” and “$x\in C$,” this means that
whenever $x\in A$ implies $x\in B$ and $x\in B$ implies $x\in C$,
then $x\in A$ implies $x\in C$. Therefore $A\subseteq C$.
