\section*{1. Determine truth values}
Throughout, \(x\) denotes an arbitrary object. We interpret \(\subset\) as ``proper subset'' and \(\subseteq\) as ``subset (not necessarily proper).''

\begin{enumerate}[label=(\alph*),leftmargin=*]

\item \(x \in \{x\}\) \quad \textbf{True.}\\
By definition, the singleton \(\{x\}\) has exactly one element, namely \(x\).

\item $\{x\} \subseteq \{x\}$ \quad \textbf{True.}\\
Every set is a subset of itself (reflexivity of $\subseteq$). Since the two sets are equal,
$\{x\}\subseteq\{x\}$ holds.

\item \(\{x\} \in \{x\}\) \quad \textbf{False.}\\
The only element of \(\{x\}\) is \(x\) (not the set \(\{x\}\) itself).

\item \(\{x\} \in \{\{x\}\}\) \quad \textbf{True.}\\
The set \(\{\{x\}\}\) has exactly one element, namely \(\{x\}\).

\item \(\varnothing \subseteq \{x\}\) \quad \textbf{True.}\\
The empty set is a subset of every set.

\item \(\varnothing \in \{x\}\) \quad \textbf{False in general (true only if \(x=\varnothing\)).}\\
If \(x\neq\varnothing\), then the sole element of \(\{x\}\) is \(x\), so \(\varnothing\notin\{x\}\).
(Example: with \(x=1\) we have \(\varnothing\notin\{1\}\).)

\end{enumerate}