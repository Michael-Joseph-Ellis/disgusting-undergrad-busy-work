\documentclass{article}
\usepackage{graphicx} % Required for inserting images
\usepackage{pdfpages}
\usepackage{hyperref}
\usepackage{bookmark}
\usepackage{setspace}
\usepackage{geometry}
\usepackage{lipsum}
\usepackage{cite}
\bibliographystyle{plain}

\title{Undergraduate Research Statement (DRAFT)}
\author{Michael Joseph Ellis}
\date{}

\begin{document}

\maketitle

\indent When acknowledging my research as multi-focal\textemdash a 
blend of Human-Computer Interaction, Artificial Intelligence, and Human Factors Psychology\textemdash I like to refer to it 
as ``Human-AI Interaction.`` As an emerging interdisciplinary field (as of the time of making this statement), my work 
explicitly considers problems in computer science and psychology, where I study the underlying meaningful value for the 
individual when AI is developed to cater to a need. Although the applications differ (and hence, two or all areas appear seemingly 
unrelated in technicality and distinct discipline), the tools, techniques, and perspectives on a problem are corollary and connect my work.

\indent For instance, consider the publications~\cite{Efficient-Vision-Transformers-for-Autonomous-Systems} 
and~\cite{AI-In-Disaster-Risk-Communication}. The first paper is on efficient vision transformers, aiming to enhance autonomous off-road 
perception by improving the computational efficiency and accuracy of vision transformers in unstructured environments. On the other hand, 
the second paper strays from a strictly technical aspect and explores how AI can be applied to risk communication regarding disasters and 
emergency decision-making. Although the applications prove the use of AI vary (given studies are of different disciplinary specialties), 
a very similar approach appears: they each show how AI techniques can extract and process critical information in dynamic and unpredictable 
environments, improving decision-making via situational awareness and response efficiency. In other words, similar fine-tuning and technique 
analysis can successfully contribute meaningfully\textemdash by showcasing real-world significance for the individual\textemdash in 
two otherwise unrelated papers.

\indent While we know how to apply AI where we can personalize experiences in controlled environments with abundant data, we know less about 
how to effectively use these methods when individual preferences and behaviors are less predictable, data is limited or inconsistent, or when 
policies and usability standards are not met/addressed. 

\indent Now, consider my research focus on addressing questions that emerge from applying AI to enhance personalized experiences 
in academia and professional settings.  

\begin{itemize}
    \item \textbf{How can we train models that adapt to individual learning styles and professional needs?} \newline 
    Many learners and professionals engage with digital platforms in unique ways, so models must detect and respond to 
    individual patterns, adjusting content delivery or task recommendations to support diverse needs effectively. 
    \item \textbf{How can AI-driven personalization provide motivation and retention in digital learning environments?} \newline 
    By identifying and responding to signs of engagement or disengagement, AI should dynamically adjust learning paths or 
    workflow recommendations, which improve motivation and reduce drop-off rates. 
    \item \textbf{How can we design models that generalize well across varied academic and professional contexts?} \newline 
    Data from different educational and professional settings can vary widely. To create adaptable and impactful AI 
    solutions, we need to find ways to train generalizable models that maintain accuracy and relevance across diverse environments, 
    data sources, and user demographics. 
\end{itemize}

\indent Currently, methods address the first challenge by brute force: to get neural networks to learn the individual's preferred learning 
style\textemdash~auditory, visual, reading/writing, and kinesthetic. Researchers monitor attention, cognitive workload, facial expressions, 
and emotional states and utilize various ML algorithms to interpret these data 
points~\cite{An-AI-based-learning-style-prediction-model-for-personalized-and-effective-learning}. However, to use these models, researchers 
have to label data themselves. Each data point needs diverse examples that show the points under uninteresting variations that the researcher 
wants the models to ignore, requiring a demanding number of samples~\cite{Deep-learning-is-combined-with-massive-scale-citizen-science}. This 
process is called supervised learning, which can be time-consuming and limits data analysis efficiency. The applicability of these methods in 
questions two and three are also directly limited by this drawback. Since prerequisite knowledge of the learning styles is to be had before 
labeling, these models are biased off topics we already know enough about, making them less suited for discovery and reliant on human 
intervention~\cite{Supervised-learning-with-decision-tree-based-methods-in-computational-and-systems-biology}.

\indent In this statement, I describe my current research, consisting of various works in AI development, Human-Centered Computing discussions, 
and decision-making based on communications of risk to an individual. Following, I discuss individual progress toward my research focus and my 
future work with connections. 

\vspace{10pt}
\begin{center}
    \section*{Current Work}
\end{center}

\begin{center}
    \item \textbf{Humans \& Technology.}
\end{center}

\indent \lipsum[1]

\begin{center}
    \item \textbf{Risk Communication \& Decision-Making.}    
\end{center}

\indent \lipsum[1]

\begin{center}
    \item \textbf{Machine Learning \& Artificial Intelligence.}
\end{center}

\indent \lipsum[1]

\vspace{10pt}
\begin{center}
    \item \section*{Individual Work}
\end{center}

Talk about individual research. i.e., independent analysis \& learning, informal research experience

\begin{itemize}
    \item analysis of papers read (add Constantinos et al. paper \cite{A-Situation-Awareness-Perspective-on-Human-AI-Interaction},  add U-Net paper \cite{ronneberger2015unetconvolutionalnetworksbiomedical} and add AI in psychiatry paper \cite{Your-Robot-Therapist-Will-See-You-Now} (look in notes)
    \item find connections between the papers or tie them to primary work \cite{A-Situation-Awareness-Perspective-on-Human-AI-Interaction}, \cite{JIANG2024100078}, \cite{shao2024exploring}
    \item no individually conducted research to talk about, unfortunately
    \item theoretical ideas I've developed (look in notes)
    \begin{itemize}
        \item explain understanding and insights
        \item take note of style and framing
    \end{itemize}
\end{itemize}

\begin{center}
    \section*{Future Work}
\end{center}

state any questions I want to explore 
maybe talk about future research 
potential approaches identified 

\begin{center}
    \bibliography{citation.bib}
\end{center}

\end{document}